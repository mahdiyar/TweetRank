\section{Conclusion}
We adapted the definition of PageRank to Twitter context where direct references among tweets (retweets or replies) are not the usual way to surf on Twitter. We proposed to take into account also mentions, followers and hashtags to determine the importance of a tweet. We presented an extension of the previous model which considers the similarity between users in terms of common hashtags. 

We have disscussed several during the previous sections the advantages and disadvantages of our model and the algorithm used to compute the TweetRank. However, the main disadvantage to our current implementation is that each time a new tweet is added, a new computation for the whole set of tweets is required. And this approach does not scale in a dynamic environment like Twitter where millions of tweets are published each second. Computing PageRank on a dynamic network is a well-studied problem. Our model could be adapted to the approaches described in \cite{Bahmani:2010:FIP:1929861.1929864,Desikan:2005:IPR:1062745.1062885} easily, but that is beyond the aim of this work.