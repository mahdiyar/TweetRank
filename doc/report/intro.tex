\section{Introduction}
Page and Brin defined the importance of a web page as the probability that at a certain time \cite{pagebmw98}, a person that randomly clicks on links will be visiting that web page. This assumption, called \emph{random surfing}, works really well on the World Wide Web since the navigation is driven by hyperlinks among web pages. 

Formally, the PageRank is the largest real eigenvector of the matrix $G$ defined as it follows:
\begin{equation}\label{eq:basic_pr}
G = \alpha R + (1-\alpha) L
\end{equation}

Each element $G_{i,j}$ represents the probability of visiting the web page $j$ given that the user is in the web page $i$. $G_{i,j}$ is defined as the weighted summation of $L_{i,j}$ and $R_{i,j}$.

$L_{i,j}$ represents the probability of visiting the web page $j$ following one of the links of the web page $i$, and $R_{i,j}$ represents the probability of randomly visiting the web page $j$ from web page $i$. $R_{i,j}$ is usually set to $\frac{1}{|W|}$ and it represents the probability of accessing the web page $j$ by a random access. 

In the world of Twitter, considering only direct references or hyperlinks among tweets is not a good approximation to the importance of a Tweet, because the direct connectivity in Twitter is sparser than the Web. Some authors realised about this problem and presented alternatives to PageRank for ranking tweets in \cite{Duan:2010:ESL:1873781.1873815,DBLP:conf/webi/NagmotiTC10}. Other authors \cite{Duan:2010:ESL:1873781.1873815,Kwak:2010:TSN:1772690.1772751,DBLP:conf/wsdm/WelchSHC11} have described how to use PageRank to rank users based on the followers, the replies, etc. However, none of them have proposed a pure PageRank-based algorithm to rank tweets. S. Ravikumar et al. presented in \cite{DBLP:journals/corr/abs-1204-0156} an approach that first computes the PageRank for users, and then distributes it among tweets. This is similar to one of the things what we do to avoid the sparsity of Twitter, however our approach embeds this into the PageRank computation itself and not as a separate step.
