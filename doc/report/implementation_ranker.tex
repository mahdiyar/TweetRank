\subsection{TweetRank}\label{sec:tweetrank_implementation}
Algorithm \ref{algo:alg1} is used to compute the TweetRank score and it is an adaptation of the classic Monte Carlo complete path stopping at dangling nodes method. The graph from Twitter $G$ is just the set of relationships described in section \ref{sec:tweetrank_definition}. Algorith \ref{algo:alg1} includes some predefined functions that are not described here, however we thought that their name is descriptive enough.

\begin{algorithm}
\caption{Compute TweetRank using MC complete path stopping at dangling nodes}
\label{algo:alg1}
{\fontsize{8}{8}\selectfont
\begin{algorithmic}
\REQUIRE Probabilities $\alpha, \beta, \gamma, \delta, \epsilon$ such that $\alpha+\beta+\gamma+\delta+\epsilon=1$, Twitter graph $G$ and $M \in \mathbb{N}$ such that $T\times M \in O(T^2)$.
\ENSURE $\pi$, TweetRank of tweets in $T(G)$.
\STATE $V_t \leftarrow 0, \forall t \in T$ \COMMENT{Visit counter for tweets}
\STATE $CP_0 \leftarrow \alpha$ \COMMENT{Cumulative probabilities for each action}
\FORALL{$t \in T(G)$}
\FOR {$m=1$ to $M$}
\STATE $ct \leftarrow t$
\STATE $stop \leftarrow False$
\WHILE {$\neg stop$}

\STATE $CP_1 \leftarrow CP_0 + \beta \cdot hasRetweetOrReply(ct)$
\STATE $CP_2 \leftarrow CP_1 + \gamma \cdot hasMentions(ct)$
\STATE $CP_3 \leftarrow CP_2 + \delta \cdot hasFriends(user(ct))$
\STATE $CP_4 \leftarrow CP_3 + \epsilon \cdot hasHashtags(ct)$

\STATE $r \leftarrow UniformRandomNumber(0, CP_4)$ \COMMENT{This ensures that all rows sums 1.0}

\IF { $r \leq CP_0$ }
\STATE $ct \leftarrow jumpToRandomTweet()$
\STATE $stop \leftarrow True$
\ELSIF { $r \leq CP_1$ }
\STATE $ct \leftarrow jumpToRetweetOrReply(ct)$
\ELSIF { $r \leq CP_2$ }
\STATE $ct \leftarrow jumpToMentionTweet(ct)$
\ELSIF { $r \leq CP_3$ }
\STATE $ct \leftarrow jumpToFriendTweet(ct)$
\ELSE
\STATE $ct \leftarrow jumpToHashtagTweet(ct)$
\ENDIF
\STATE $V_{ct} \leftarrow V_{ct} + 1$ 
\ENDWHILE
\ENDFOR
\ENDFOR
\STATE $\pi_t \leftarrow Normalize(V)$ \COMMENT{TweetRank as the normalized visit vector}
\end{algorithmic}}
\end{algorithm}

Observe that the running time of algorithm \ref{algo:alg1} is non-deterministic, since it depends on the structure of the graph itself and the $\alpha$ parameter. Assuming random access memory, and no dangling nodes in graph, the previous algorithm has an expected running time bounded by $O(|T| \times M \times \mathbb{E}[|w|])$, where $\mathbb{E}[|w|]$ is the expected length of the random walk. The expected length of the random walk is obtained from the following
equation:
\begin{equation}
\mathbb{E}[|w|] = \sum_{k=1}^{\infty} P(|w| = k) \cdot k = \sum_{k=1}^{\infty} (1 - \alpha)^{k-1} \cdot \alpha \cdot k = \frac{1}{\alpha}
\end{equation}

Given that $|T| \times M$ must be $O(|T|^ 2)$ to ensure a good approximation of TweetRank, the expected running time of algorithm \ref{algo:alg1} is:
\begin{equation}\label{eq:running_time}
O(|T| \times M \times \mathbb{E}[|w|]) = O(|T|^2 \times \frac{1}{\alpha}) = O(|T|^2)
\end{equation}

Note that equation \ref{eq:running_time} is an upper bound on the expected running time if the graph contains dangling nodes. On the other side, the assumption of random access memory might be a problem for a large index which does not fit in the main memory of a single machine. However, implementing TweetRank on a large-scale distributed system is not the approach of this work and we will made this assumption to keep the analysis simple.